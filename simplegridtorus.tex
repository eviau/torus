\documentclass{standalone}

%%%%%%%%%%%%%%%%%%%%%%%%%%%%%%%% variables

%% grid definition
% x : number of vertical lines (length of x-axis)
% y : number of horizontal lines (length of y-axis)
\newcommand\x{6}
\newcommand\y{5}

%% blank background definition
% x : horizontal length of the background.
% y : vertical length of the background. 
\newcommand\marginx{7}
\newcommand\marginy{6}

%%%%%%%%%%%%%%%%%%%%%%%%%%%%%%%%%%%%%%%%%%%%%%%%%

\usepackage[french]{babel}
\usepackage[utf8]{inputenc}
% %Verifier que la locale est bien: "fr_FR.UTF-8"
\usepackage[T1]{fontenc}

\usepackage{tikz}
\usetikzlibrary{decorations.markings,arrows}
\usetikzlibrary{matrix}

\begin{document}
\begin{tikzpicture}
\begin{scope}
\draw (-1,-1) rectangle (\marginx,\marginy);
\begin{scope}[very thick,decoration={
    markings,
    mark=at position 0.5 with {\arrow{latex}}}
    ] 
    
    \draw[postaction={decorate}] (0,0)--(\x,0);
    \draw[postaction={decorate}] (\x,0)--(\x,\y);
    \draw[postaction={decorate}] (0,\y)--(\x,\y);
    \draw[postaction={decorate}] (0,0)--(0,\y);

\end{scope}

    \begin{scope}[very thick,decoration={
    markings,
    mark=at position 0.55 with {\arrow{latex}}}
    ] 
    \draw[postaction={decorate}] (\x,0)--(\x,\y);
    \draw[postaction={decorate}] (0,0)--(0,\y);

\end{scope}
\begin{scope}
\draw [step=1cm](0,0) grid (\x,\y);
\end{scope}

\end{scope}


\end{tikzpicture}
\end{document}